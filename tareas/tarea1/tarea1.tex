\documentclass[12pt]{article}

% --- Página y tipografía ---
\usepackage[letterpaper,margin=2.5cm]{geometry}
\usepackage[T1]{fontenc}
\usepackage[utf8]{inputenc} % si compilas con pdfLaTeX
\usepackage{lmodern}
\usepackage{microtype}
\usepackage[spanish]{babel}
\usepackage{setspace}
\doublespacing

% --- Imágenes y color ---
\usepackage{graphicx}
\usepackage{xcolor}

% --- Control fino de espacios ---
\usepackage{setspace}
\setlength{\parindent}{0pt}

\begin{document}
\thispagestyle{empty}

% ===== Encabezado con logos + texto =====
\begin{minipage}[c]{0.18\textwidth}
    \centering
    % Cambia por tu logo izquierdo
    \includegraphics[width=0.95\linewidth]{img/logo_usac.jpeg}
\end{minipage}
\hfill
\begin{minipage}[c]{0.60\textwidth}
    \small
    Universidad de San Carlos de Guatemala\\
    Escuela de Ciencias Físicas y Matemáticas\\
    Nombre: Diego Pablo Vásquez Vásquez\\
    Carnet: 202500362\\
    Programación 1\\
\end{minipage}
\hfill
\begin{minipage}[c]{0.18\textwidth}
    \centering
    % Cambia por tu logo derecho
    \includegraphics[width=1.4\linewidth]{img/logo_ecfm.jpg}
\end{minipage}

\vspace{0.5cm}

% Línea horizontal superior (gruesa)
\noindent\rule{\textwidth}{1.2pt}

\vspace{0.2cm}

% ===== Título =====
\begin{center}
    {\Large\scshape Área de investigación como matemático}\\[0.3em]
\end{center}

\vspace{0.1cm}

% Fecha
\begin{center}
    \small\scshape \today
\end{center}

\vspace{0.2cm}

% Línea horizontal inferior (gruesa)
\noindent\rule{\textwidth}{1.2pt}

\vspace{0.6cm}

% ===== Caja de resumen =====
\noindent
\colorbox{gray!35}{%
    \parbox{\textwidth}{%
        \vspace{0.6em}
        \textbf{Resumen}\\[0.3em]
        \small
			Como matemático mis dos objetivos principales son los siguientes:
			\begin{enumerate}
				\item Aplicar la matemática al estudio de las aplicaciones de informática teórica.
				\item Mejorar la educación matemática en Guatemala.
			\end{enumerate}
        \vspace{0.8em}
    }%
}


\section*{Computación, Educación y Matemática}
	En el siglo XX, especialmente durante la Segunda Guerra Mundial (1939 -- 1945), la matemática jugó un papel importante en el desarrollo de la computación moderna tal y como la conocemos actualmente. El matemático inglés Alan M. Turing junto a un equipo de científicos desarrollaron la máquina capaz de descifrar los mensajes de la Alemania Nazi y así las fuerzas navales de Inglaterra podrían adelantarse a los hechos. Esta máquina llamada <<Enigma>> fue la primera concepción física de una computadora moderna. 
	
	En los años 50, el matemático austrohúngaro John Von Neumann diseñó el primer prototipo de la arquitectura de una computadora. En la cual se contemplaban los periféricos de entrada salida, la Unidad Central de Procesamiento (CPU) conformada por la Unidad Aritmético-Lógica (ALU), la Unidad de Control (UC), contador de programa y Memoria Principal y que es hasta el día de hoy la base para el diseño de las computadoras modernas. 
	
	Todo este trabajo tras bastidores provocó el inicio de las tecnologías modernas, especialmente, de la computación. La matemática ha tenido una fuerte influencia en el desarrollo de la computación teórica y aplicada. Incluso me atrevo a decir que la Computación, perse, es una rama de la Matemática aplicada, la Filosofía aplicada y la Física aplicada. 
	
	Como matemático mi especialidad al graduarme es dedicarme al trabajo de la computación teórica, buscando una respuesta a este estudio me incursioné en el estudio de la Ingeniería en Sistemas y aunque satisfizo parte de mis conocimientos, sé que como matemático aplicaré todas las estructuras matemáticas y disciplinas afines para la optimización de algoritmos, desarrollo de la I.A., aplicaciones en educación y también mi otra rama de especialización como Profesor de Enseñanza Media adicionando mis otros dos ramos (Sistemas y Matemática) es mejorar el campo de la educación matemática en Guatemala como una fuente del pensamiento crítico.
%
%\section{Marco Teórico}
%
%\section{Diseño Experimental}
%
%\section{Resultados}
%
%\section{Discusión de Resultados}
%
%\section{Conclusiones}
%
%\section{Referencias}
%
%\section{Anexos}
\end{document}

\end{document}
